%%%%%%%%%%%%%%%%%%%%%%%%%%%%%%%%%%%%%%%%%
% University Assignment Title Page 
% LaTeX Template
% Version 1.0 (27/12/12)
%
% This template has been downloaded from:
% http://www.LaTeXTemplates.com
%
% Original author:
% WikiBooks (http://en.wikibooks.org/wiki/LaTeX/Title_Creation)
%
% License:
% CC BY-NC-SA 3.0 (http://creativecommons.org/licenses/by-nc-sa/3.0/)
% 
% Instructions for using this template:
% This title page is capable of being compiled as is. This is not useful for 
% including it in another document. To do this, you have two options: 
%
% 1) Copy/paste everything between \begin{document} and \end{document} 
% starting at \begin{titlepage} and paste this into another LaTeX file where you 
% want your title page.
% OR
% 2) Remove everything outside the \begin{titlepage} and \end{titlepage} and 
% move this file to the same directory as the LaTeX file you wish to add it to. 
% Then add %%%%%%%%%%%%%%%%%%%%%%%%%%%%%%%%%%%%%%%%%
% University Assignment Title Page 
% LaTeX Template
% Version 1.0 (27/12/12)
%
% This template has been downloaded from:
% http://www.LaTeXTemplates.com
%
% Original author:
% WikiBooks (http://en.wikibooks.org/wiki/LaTeX/Title_Creation)
%
% License:
% CC BY-NC-SA 3.0 (http://creativecommons.org/licenses/by-nc-sa/3.0/)
% 
% Instructions for using this template:
% This title page is capable of being compiled as is. This is not useful for 
% including it in another document. To do this, you have two options: 
%
% 1) Copy/paste everything between \begin{document} and \end{document} 
% starting at \begin{titlepage} and paste this into another LaTeX file where you 
% want your title page.
% OR
% 2) Remove everything outside the \begin{titlepage} and \end{titlepage} and 
% move this file to the same directory as the LaTeX file you wish to add it to. 
% Then add %%%%%%%%%%%%%%%%%%%%%%%%%%%%%%%%%%%%%%%%%
% University Assignment Title Page 
% LaTeX Template
% Version 1.0 (27/12/12)
%
% This template has been downloaded from:
% http://www.LaTeXTemplates.com
%
% Original author:
% WikiBooks (http://en.wikibooks.org/wiki/LaTeX/Title_Creation)
%
% License:
% CC BY-NC-SA 3.0 (http://creativecommons.org/licenses/by-nc-sa/3.0/)
% 
% Instructions for using this template:
% This title page is capable of being compiled as is. This is not useful for 
% including it in another document. To do this, you have two options: 
%
% 1) Copy/paste everything between \begin{document} and \end{document} 
% starting at \begin{titlepage} and paste this into another LaTeX file where you 
% want your title page.
% OR
% 2) Remove everything outside the \begin{titlepage} and \end{titlepage} and 
% move this file to the same directory as the LaTeX file you wish to add it to. 
% Then add %%%%%%%%%%%%%%%%%%%%%%%%%%%%%%%%%%%%%%%%%
% University Assignment Title Page 
% LaTeX Template
% Version 1.0 (27/12/12)
%
% This template has been downloaded from:
% http://www.LaTeXTemplates.com
%
% Original author:
% WikiBooks (http://en.wikibooks.org/wiki/LaTeX/Title_Creation)
%
% License:
% CC BY-NC-SA 3.0 (http://creativecommons.org/licenses/by-nc-sa/3.0/)
% 
% Instructions for using this template:
% This title page is capable of being compiled as is. This is not useful for 
% including it in another document. To do this, you have two options: 
%
% 1) Copy/paste everything between \begin{document} and \end{document} 
% starting at \begin{titlepage} and paste this into another LaTeX file where you 
% want your title page.
% OR
% 2) Remove everything outside the \begin{titlepage} and \end{titlepage} and 
% move this file to the same directory as the LaTeX file you wish to add it to. 
% Then add \input{./title_page_1.tex} to your LaTeX file where you want your
% title page.
%
%%%%%%%%%%%%%%%%%%%%%%%%%%%%%%%%%%%%%%%%%
%\title{Title page with logo}
%----------------------------------------------------------------------------------------
%	PACKAGES AND OTHER DOCUMENT CONFIGURATIONS
%----------------------------------------------------------------------------------------

\documentclass[12pt]{article}
\usepackage[english]{babel}
\usepackage[utf8x]{inputenc}
\usepackage{amsmath}
\usepackage{graphicx}
\usepackage{listings}
\usepackage{color}
\usepackage{xcolor}
\usepackage{amssymb}
\usepackage{hyperref}

\definecolor{dkgreen}{rgb}{0,0.6,0}
\definecolor{gray}{rgb}{0.5,0.5,0.5}
\definecolor{mauve}{rgb}{0.58,0,0.82}

\lstset{frame=tb,
	language=C,
	aboveskip=3mm,
	belowskip=3mm,
	showstringspaces=false,
	columns=flexible,
	basicstyle={\small\ttfamily},
	numbers=none,
	numberstyle=\tiny\color{gray},
	keywordstyle=\color{blue},
	commentstyle=\color{dkgreen},
	stringstyle=\color{mauve},
	breaklines=true,
	breakatwhitespace=true,
	tabsize=3
}

\makeatletter
\newsavebox\myboxA
\newsavebox\myboxB
\newlength\mylenA

\newcommand*\xoverline[2][0.75]{%
	\sbox{\myboxA}{$\m@th#2$}%
	\setbox\myboxB\null% Phantom box
	\ht\myboxB=\ht\myboxA%
	\dp\myboxB=\dp\myboxA%
	\wd\myboxB=#1\wd\myboxA% Scale phantom
	\sbox\myboxB{$\m@th\overline{\copy\myboxB}$}%  Overlined phantom
	\setlength\mylenA{\the\wd\myboxA}%   calc width diff
	\addtolength\mylenA{-\the\wd\myboxB}%
	\ifdim\wd\myboxB<\wd\myboxA%
	\rlap{\hskip 0.5\mylenA\usebox\myboxB}{\usebox\myboxA}%
	\else
	\hskip -0.5\mylenA\rlap{\usebox\myboxA}{\hskip 0.5\mylenA\usebox\myboxB}%
	\fi}
\makeatother

\begin{document}

\begin{titlepage}

\newcommand{\HRule}{\rule{\linewidth}{0.5mm}} % Defines a new command for the horizontal lines, change thickness here

\center % Center everything on the page
 
%----------------------------------------------------------------------------------------
%	HEADING SECTIONS
%----------------------------------------------------------------------------------------

\textsc{\LARGE Politecnico di Milano}\\[1.5cm] % Name of your university/college
\textsc{\Large Dipartimento Elettronica, Informazione e Bioingegneria}\\[0.5cm] % Major heading such as course name
\textsc{\large HEAPLab Project Report}\\[0.5cm] % Minor heading such as course title

%----------------------------------------------------------------------------------------
%	TITLE SECTION
%----------------------------------------------------------------------------------------

\HRule \\[0.4cm]
{ \huge \bfseries Floating Point Predictability}\\[0.4cm] % Title of your document
\HRule \\[1.5cm]
 
%----------------------------------------------------------------------------------------
%	AUTHOR SECTION
%----------------------------------------------------------------------------------------

\begin{minipage}{0.4\textwidth}
\begin{flushleft} \large
\emph{Authors:}\\
Alessio \textsc{Cantina}\\ % Your name
Simone \textsc{Crippa}
\end{flushleft}
\end{minipage}
~
\begin{minipage}{0.5\textwidth}
\begin{flushright} \large
\emph{Supervisor:} \\
Dr. Federico \textsc{Reghenzani} % Supervisor's Name
\end{flushright}
\end{minipage}\\[1cm]

% If you don't want a supervisor, uncomment the two lines below and remove the section above
%\Large \emph{Author:}\\
%John \textsc{Smith}\\[3cm] % Your name

%----------------------------------------------------------------------------------------
%	DATE SECTION
%----------------------------------------------------------------------------------------

{\large \today}\\[2cm] % Date, change the \today to a set date if you want to be precise

%----------------------------------------------------------------------------------------
%	LOGO SECTION
%----------------------------------------------------------------------------------------

\includegraphics[width=100pt]{heaplogo.pdf}\\[1cm] % Include a department/university logo - this will require the graphicx package
 
%----------------------------------------------------------------------------------------

\vfill % Fill the rest of the page with whitespace

\end{titlepage}




\begin{abstract}
\textcolor{red}{Your abstract. - Summarize your work.}

This project aims to investigate floating point vs fixed point predictability.

To achieve this result, we have used 6 Worst Case Execution Time (WCET) benchmarks which involve a large number of floating point instructions.
We converted these benchmarks using only fixed point variables and then we added the logic to obtain multiple measurements.

Finally, we computed and analysed 8 statistics to find out if floating point operations or fixed point operations are more predictable.

\end{abstract}

\section{Introduction}

\textcolor{red}{Your introduction goes here! Some examples of commonly used commands and features are listed below, to help you get started.}

In this paragraph we describe all the commands used to process the benchmarks.

\subsection{Devices used}

To have a better view of the problem, we decided to get samples from devices with different architecture.
In this project, we considered 3 architectures and 5 devices.\newline 
armv7l devices:
\begin{itemize}
	\item Device name: Freescale i.MX 6 Quad. Processor configuration: 4x Cortex-A9 @ 1.2 GHz
	\item Device name: ODROID-XU3. Processor name: Exynos 5422. Processor configuration: big.LITTLE 8 cores, 4x Cortex-A15 @ 2.0 GHz, 4x Cortex-A7 @ 1.4 GHz 
\end{itemize}
armv8a devices:
\begin{itemize}
	\item Device name: Xiaomi Mi 5s. Processor name: Snapdragon 821 (MSM8996). Processor configuration: big.LITTLE 4 cores: 2x Kryo @ 2.15 GHz, 2x Kryo @ 1.6 GHz
	\item Device name: Xiaomi Mi Mix 2. Processor name: Snapdragon 835 (MSM8998). Processor configuration: big.LITTLE 8 cores: 4x Kryo @ 2.45 GHz, 4x Kryo @ 1.9 GHz
\end{itemize}
x86\_64 devices:
\begin{itemize}
	\item Device name: Dell XPS 13 9360. Processor name: Intel i5 8250u. Processor configuration: 4 cores (8 threads) @ 3.40 GHz
\end{itemize}
\subsection{Compiling commands}

\begin{itemize}
	\item Compile a floating point benchmark with the support library: \begin{verbatim}gcc -o "output_name" "benchmark_name"\end{verbatim}
	\item Cross-compile a benchmark, useful to execute a benchmark on a smartphone with ARM processor: \begin{verbatim}arm-linux-gnueabi-gcc -static -march="arm_platform" -o 
	"output_name" "benchmark_name"\end{verbatim}
	\item To compile a fixed point benchmark remember to append at the end of the command: \begin{verbatim}fixed_op_64bit.c\end{verbatim}
\end{itemize}


\subsection{make commands}

A makefile has been produced so that make can be used to compile and execute all the benchmarks at once:

\begin{itemize}
	\item Initialize the project directory with support folders:\begin{verbatim}make init\end{verbatim}
	\item Compile all the benchmarks: \begin{verbatim}make\end{verbatim}
	\item Execute all the benchmarks: \begin{verbatim}make run\end{verbatim}
	\item Clear the project directory: \begin{verbatim}make clean\end{verbatim}
\end{itemize}

Note: the makefile supports gcc compilation as is, but it needs to be modified for cross compilation. Detailed instructions are available inside the makefile.

\subsection{Additional commands}

These commands are used to run benchmarks with a stress test running on the CPU cache

\begin{itemize}
	\item Stress the CPU cache on a specific set of CPUs:\begin{verbatim}taskset -c cpu_list stress-ng --cache N \end{verbatim} where cpu\_list is the select cluster (start\_CPU - end\_CPU) and N is the number of worker threads (1 thread for each CPU core)
	\item Execute the benchmarks on a specific set of CPUs: \begin{verbatim}taskset -c cpu_list\end{verbatim}
\end{itemize}

\section{Design and Implementation}

\subsection{Benchmarks}

To investigate this issue, we selected 6 floating point benchmarks from \href{http://www.mrtc.mdh.se/projects/wcet/benchmarks.html}{mrtc.mdh.se}.
These are the selected benchmarks:
\begin{enumerate}
	\item ludcmp: implementation of the \href{https://en.wikipedia.org/wiki/LU_decomposition}{LU decomposition} algorithm
	\item minver: inversion of a floating point matrix
	\item qsort: non-recursive implementation of quick sort algorithm
	\item qurt: root computation of quadratic equations
	\item select: function to select the ${N}^{th}$ largest number in a floating point array
	\item sqrt: square root function implemented by Taylor series
\end{enumerate}

Since there is no automatic way in C to convert floating point operations into fixed point, we have defined rules to convert floating point benchmarks into fixed points.
\begin{enumerate}
	\item Convert all the floating values into fixed point values in a fair way, keeping the same variable dimension:
	\begin{itemize}
		\item float $\rightarrow$ int32\_t	
		\item double $\rightarrow$ int64\_t
	\end{itemize}
	\item Multiply every floating point value by ${2}^{SHIFT\_AMOUNT}$.
	\item Convert every multiplication and division into the corresponding fixed point function
	\begin{itemize}
		\item multiplication $\rightarrow$ fixed\_mul\_64bit	
		\item division $\rightarrow$ fixed\_div\_64bit
	\end{itemize}
\end{enumerate}

Moreover, all benchmarks are executed on random generated numbers, except qurt which computes multiple times the inversion of a matrix. 
\subsection{Sampling of the benchmarks}

Original benchmarks have no built-in method to obtain the elapsed execution time. We have implemented and integrated in C a function which randomly generates arguments for the benchmark and measures the elapsed time.
Measurements are stored in a .txt files for future analysis.

This is an example of the main function:
\begin{itemize}
	\item srand() is used to generate random numbers, using a seed to keep track of generated numbers
	\item EXEC\_NUM is a constant which regulates the number of samples generated. In our case, we generate 100.000 samples.
	\item function\_to\_call(val) is the benchmark function with the argument.
	\item diff(timespec,timespec) is a support function which makes the difference between two time instants.
	\item CLOCK\_MONOTONIC\_RAW is used to get the time instant from the system without alterations (even from ntp)
	\item SHIFT\_AMOUNT is used to convert floating point values into fixed points, an higher SHIFT\_AMOUNT means an higher precision.
\end{itemize}
\begin{lstlisting}
int main()
{
	struct timespec start,end;
	FILE * fp;
	fp = fopen ("benchmark_results.txt","w");
	int64_t val;
	srand(5);
	for (int i=0; i< EXEC_NUM ; i++){
		//convert float to int64 only if the benchmark is fixed point, if not no conversion is done
		val = (int64_t)((((rand() % 10000) / 100) - 50) * pow(2,SHIFT_AMOUNT));
		
		clock_gettime(CLOCK_MONOTONIC_RAW, &start);
		function_to_call(val);
		clock_gettime(CLOCK_MONOTONIC_RAW, &end);
		
		fprintf (fp, "%lld\n",(long long)(diff(start,end).tv_sec * pow(10,9))+(long long)diff(start,end).tv_nsec);
	}
	
	fclose (fp);
	return 0;
}
\end{lstlisting}

\subsection{Fixed point operations}

To support fixed point operations we had to introduce a shift amount, so that floating point numbers can be represented as integers with a fixed precision, given by the shift amount.\newline
Since we have analysed different architectures we had to take into account the limitations of the selected architectures.
x86\_64 have the support for 128-bit variables, which are necessary to compute multiplications and divisions. Instead, ARM architectures have no support for 128-bit variables, so we implemented a support library to handle this discrepancy.
The library automatically select the operations by checking if the architecture support 128-bit integers, if there is no support, the benchmark will use a portable implementation of multiplication and division which are based on 64-bit integers.\newline
Our implementation of fixed division and multiplication works with a variable shift amount, in our benchmarks we have always used a shift amount of 30 because in our scenario it allows to have the necessary precision.\newline
Note: to guarantee fairness we use hard-coded values for constants in benchmarks that are shifted by 30-bits, even though operations supports dynamical shift amounts these values have to be modified to comply with the new selected shift amount.\newline

\textbf{fixed\_div for 128-bit support}

\begin{lstlisting}
int64_t fixed_div_64(int64_t x, int64_t y, int shift_amount)
{
    return ((((__int128)x << shift_amount) / y));
}
\end{lstlisting}

\textbf{fixed\_mul for 128-bit support}

\begin{lstlisting}
int64_t fixed_mul_64(int64_t x, int64_t y, int shift_amount)
{
	return (int64_t)((((__int128)x * (__int128)y)) >> shift_amount);
}
\end{lstlisting}

\textbf{fixed\_div without 128-bit support}\newline
To implement the divison mantaining the 128 bit precision needed during the divison operation,  the dividend is splitted into two 64 bit integers. Then the divison is perfomed using the technique that can be found at this \href{https://codereview.stackexchange.com/questions/67962/mostly-portable-128-by-64-bit-division}{link}.

\textbf{fixed\_mul without 128-bit support}\newline
To implement the multiplication mantaining the 128 bit precision needed during the multiplication operation,  the two factors are splitted into two 64 bit integers each. Then the multiplication is perfomed using the technique that can be found at this \href{https://stackoverflow.com/questions/31652875/fastest-way-to-multiply-two-64-bit-ints-to-128-bit-then-to-64-bit}{link}.



\subsection{Statistics used}
To compare floating point results to fixed point results we computed through a Python script 8 metrics useful to analyse the predictability of a benchmark.

\begin{enumerate}
	\item Minimum execution time
	\item Maximum execution time
	\item Sample mean, which is computed as: $$\xoverline[0.8]{x} = \frac{X_1 + X_2 + \cdots + X_n}{n}
	= \frac{1}{n}\sum_{i}^{n} X_i$$
	\item Sample variance, which is computed as:$${s}^{2} = \frac{\sqrt{\sum_{i}^{n} {(x_i - \xoverline[0.8]{x})}^{2}}}{n-1}$$
	\item Sample standard deviation, which is computed as: 
	$$s = \frac{\sqrt{\sum_{i}^{n} (x_i - \xoverline[0.8]{x})}}{n-1}$$
	\item KPSS test statistic, which is computed as:$$S = \frac{{T}^{-2}\sum_{t=1}^{T} {\hat{S}_t}^{2}}{\sigma_\epsilon}$$
	Where:
	\begin{itemize}
		\item T is the sample size
		\item ${\sigma_\epsilon}$ is the long-run variance
		\item ${\hat{S}_t} = \sum_{i}^{t}e(t)$ is the partial sum of the errors of the regression $y(t)$
	\end{itemize}
	\item BDS test statistic, which is computed as: $$BDS_{\epsilon,m} = \frac{\sqrt{N}[C_{\epsilon,m}-{(C_{\epsilon,1})}^{m}]}{\sqrt{V_{\epsilon,m}}}$$
	Where:
	\begin{itemize}
		\item $C_{\epsilon,m} = \frac{1}{N_m(N_m-1)}\sum_{i\not=j}{I_{i,j;\epsilon}}$\\[0.1cm]
		Where:
		\begin{itemize}
			\item m is the number of embedding dimensions, used to embed the time series into m-dimensional vectors.
			\item ${I_{i,j;\epsilon}} = 1 \;\; if ||{x}^{m}_i - {x}^{m}_j|| <= \epsilon  \;\; = 0 \;\; otherwise$
		\end{itemize}
		
		\item $V_{\epsilon,m} = 4[{K}^{m} + 2\sum_{j=1}^{m-1}{K}^{m-j}{C}^{2j}_\epsilon + {(m-1)}^{2}{C}^{2m}_\epsilon - {m}^{2}K{C}^{2m-2}_\epsilon]$
		Where:
		\begin{itemize}
			\item $K=K_\epsilon = \frac{6}{N_m(N_m-1)*(N_m-2)}\sum_{i<j<N}h_{i,j,N;\epsilon}$
			\item $h_{i,j,N;\epsilon} = \frac{[I_{i,j;\epsilon}I_{j,N;\epsilon}+I_{i,N;\epsilon}I_{N,j;\epsilon}+I_{j,i;\epsilon}I_{i,N;\epsilon}]}{3}$
		\end{itemize}
	\end{itemize}
	We know under some hypothesis, that the quantity $[C_{\epsilon,m}-{(C_{\epsilon,1})}^{m}]$ can be considered as an asymptotic normal distribution with zero mean and variance $V_{\epsilon,m}$
	\item Hurst exponent, which is computed as:
	$\operatorname{E} \left [ \frac{R(n)}{S(n)} \right ]=C n^H  \text{  as } n \to \infty  \, $
	Where:
	\begin{itemize}
		\item $R(n)$ is the range of the first n cumulative deviations from the mean, and $s(n)$ is their standard deviation
		\item $\operatorname{E} \left [x \right ] \,$ is the expected value
		\item n is the time span of the observation (number of data points in a time series)
		\item C is a constant.
	\end{itemize}
\end{enumerate}

We are interested in execution time predictability, so we need to compare how much the measurements are spread through sample standard deviation and sample variance.
\href{http://debis.deu.edu.tr/userweb//onder.hanedar/dosyalar/kpss.pdf}{KPSS} test statistic is a one tailed test, which indicates if the time series is stationary around a mean or a linear trend. A stationary time series have constant statistical properties (like mean and variance) over time.
\begin{itemize}
	\item Null hypothesis: the data is stationary
	\item Alternative hypothesis: the data is not stationary
\end{itemize}
Instead, \href{https://www.researchgate.net/publication/46554708_A_Fast_Algorithm_for_the_BDS_Statistic}{BDS} statistic is a double tailed test, which indicates if the time series is serial dependent.
\begin{itemize}
 	\item Null hypothesis: the data is independently and identically distributed (I.I.D.)
 	\item Alternative hypothesis: the data is not I.I.D.
\end{itemize}

Since BDS test statistic complexity increase with the sample size, it can't be done on all the samples at once, instead we compute it using a sliding window on the samples and then select the maximum BDS obtained.



\section{Experimental Results}

\section{Conclusions}

\end{document} to your LaTeX file where you want your
% title page.
%
%%%%%%%%%%%%%%%%%%%%%%%%%%%%%%%%%%%%%%%%%
%\title{Title page with logo}
%----------------------------------------------------------------------------------------
%	PACKAGES AND OTHER DOCUMENT CONFIGURATIONS
%----------------------------------------------------------------------------------------

\documentclass[12pt]{article}
\usepackage[english]{babel}
\usepackage[utf8x]{inputenc}
\usepackage{amsmath}
\usepackage{graphicx}
\usepackage{listings}
\usepackage{color}
\usepackage{xcolor}
\usepackage{amssymb}
\usepackage{hyperref}

\definecolor{dkgreen}{rgb}{0,0.6,0}
\definecolor{gray}{rgb}{0.5,0.5,0.5}
\definecolor{mauve}{rgb}{0.58,0,0.82}

\lstset{frame=tb,
	language=C,
	aboveskip=3mm,
	belowskip=3mm,
	showstringspaces=false,
	columns=flexible,
	basicstyle={\small\ttfamily},
	numbers=none,
	numberstyle=\tiny\color{gray},
	keywordstyle=\color{blue},
	commentstyle=\color{dkgreen},
	stringstyle=\color{mauve},
	breaklines=true,
	breakatwhitespace=true,
	tabsize=3
}

\makeatletter
\newsavebox\myboxA
\newsavebox\myboxB
\newlength\mylenA

\newcommand*\xoverline[2][0.75]{%
	\sbox{\myboxA}{$\m@th#2$}%
	\setbox\myboxB\null% Phantom box
	\ht\myboxB=\ht\myboxA%
	\dp\myboxB=\dp\myboxA%
	\wd\myboxB=#1\wd\myboxA% Scale phantom
	\sbox\myboxB{$\m@th\overline{\copy\myboxB}$}%  Overlined phantom
	\setlength\mylenA{\the\wd\myboxA}%   calc width diff
	\addtolength\mylenA{-\the\wd\myboxB}%
	\ifdim\wd\myboxB<\wd\myboxA%
	\rlap{\hskip 0.5\mylenA\usebox\myboxB}{\usebox\myboxA}%
	\else
	\hskip -0.5\mylenA\rlap{\usebox\myboxA}{\hskip 0.5\mylenA\usebox\myboxB}%
	\fi}
\makeatother

\begin{document}

\begin{titlepage}

\newcommand{\HRule}{\rule{\linewidth}{0.5mm}} % Defines a new command for the horizontal lines, change thickness here

\center % Center everything on the page
 
%----------------------------------------------------------------------------------------
%	HEADING SECTIONS
%----------------------------------------------------------------------------------------

\textsc{\LARGE Politecnico di Milano}\\[1.5cm] % Name of your university/college
\textsc{\Large Dipartimento Elettronica, Informazione e Bioingegneria}\\[0.5cm] % Major heading such as course name
\textsc{\large HEAPLab Project Report}\\[0.5cm] % Minor heading such as course title

%----------------------------------------------------------------------------------------
%	TITLE SECTION
%----------------------------------------------------------------------------------------

\HRule \\[0.4cm]
{ \huge \bfseries Floating Point Predictability}\\[0.4cm] % Title of your document
\HRule \\[1.5cm]
 
%----------------------------------------------------------------------------------------
%	AUTHOR SECTION
%----------------------------------------------------------------------------------------

\begin{minipage}{0.4\textwidth}
\begin{flushleft} \large
\emph{Authors:}\\
Alessio \textsc{Cantina}\\ % Your name
Simone \textsc{Crippa}
\end{flushleft}
\end{minipage}
~
\begin{minipage}{0.5\textwidth}
\begin{flushright} \large
\emph{Supervisor:} \\
Dr. Federico \textsc{Reghenzani} % Supervisor's Name
\end{flushright}
\end{minipage}\\[1cm]

% If you don't want a supervisor, uncomment the two lines below and remove the section above
%\Large \emph{Author:}\\
%John \textsc{Smith}\\[3cm] % Your name

%----------------------------------------------------------------------------------------
%	DATE SECTION
%----------------------------------------------------------------------------------------

{\large \today}\\[2cm] % Date, change the \today to a set date if you want to be precise

%----------------------------------------------------------------------------------------
%	LOGO SECTION
%----------------------------------------------------------------------------------------

\includegraphics[width=100pt]{heaplogo.pdf}\\[1cm] % Include a department/university logo - this will require the graphicx package
 
%----------------------------------------------------------------------------------------

\vfill % Fill the rest of the page with whitespace

\end{titlepage}




\begin{abstract}
\textcolor{red}{Your abstract. - Summarize your work.}

This project aims to investigate floating point vs fixed point predictability.

To achieve this result, we have used 6 Worst Case Execution Time (WCET) benchmarks which involve a large number of floating point instructions.
We converted these benchmarks using only fixed point variables and then we added the logic to obtain multiple measurements.

Finally, we computed and analysed 8 statistics to find out if floating point operations or fixed point operations are more predictable.

\end{abstract}

\section{Introduction}

\textcolor{red}{Your introduction goes here! Some examples of commonly used commands and features are listed below, to help you get started.}

In this paragraph we describe all the commands used to process the benchmarks.

\subsection{Devices used}

To have a better view of the problem, we decided to get samples from devices with different architecture.
In this project, we considered 3 architectures and 5 devices.\newline 
armv7l devices:
\begin{itemize}
	\item Device name: Freescale i.MX 6 Quad. Processor configuration: 4x Cortex-A9 @ 1.2 GHz
	\item Device name: ODROID-XU3. Processor name: Exynos 5422. Processor configuration: big.LITTLE 8 cores, 4x Cortex-A15 @ 2.0 GHz, 4x Cortex-A7 @ 1.4 GHz 
\end{itemize}
armv8a devices:
\begin{itemize}
	\item Device name: Xiaomi Mi 5s. Processor name: Snapdragon 821 (MSM8996). Processor configuration: big.LITTLE 4 cores: 2x Kryo @ 2.15 GHz, 2x Kryo @ 1.6 GHz
	\item Device name: Xiaomi Mi Mix 2. Processor name: Snapdragon 835 (MSM8998). Processor configuration: big.LITTLE 8 cores: 4x Kryo @ 2.45 GHz, 4x Kryo @ 1.9 GHz
\end{itemize}
x86\_64 devices:
\begin{itemize}
	\item Device name: Dell XPS 13 9360. Processor name: Intel i5 8250u. Processor configuration: 4 cores (8 threads) @ 3.40 GHz
\end{itemize}
\subsection{Compiling commands}

\begin{itemize}
	\item Compile a floating point benchmark with the support library: \begin{verbatim}gcc -o "output_name" "benchmark_name"\end{verbatim}
	\item Cross-compile a benchmark, useful to execute a benchmark on a smartphone with ARM processor: \begin{verbatim}arm-linux-gnueabi-gcc -static -march="arm_platform" -o 
	"output_name" "benchmark_name"\end{verbatim}
	\item To compile a fixed point benchmark remember to append at the end of the command: \begin{verbatim}fixed_op_64bit.c\end{verbatim}
\end{itemize}


\subsection{make commands}

A makefile has been produced so that make can be used to compile and execute all the benchmarks at once:

\begin{itemize}
	\item Initialize the project directory with support folders:\begin{verbatim}make init\end{verbatim}
	\item Compile all the benchmarks: \begin{verbatim}make\end{verbatim}
	\item Execute all the benchmarks: \begin{verbatim}make run\end{verbatim}
	\item Clear the project directory: \begin{verbatim}make clean\end{verbatim}
\end{itemize}

Note: the makefile supports gcc compilation as is, but it needs to be modified for cross compilation. Detailed instructions are available inside the makefile.

\subsection{Additional commands}

These commands are used to run benchmarks with a stress test running on the CPU cache

\begin{itemize}
	\item Stress the CPU cache on a specific set of CPUs:\begin{verbatim}taskset -c cpu_list stress-ng --cache N \end{verbatim} where cpu\_list is the select cluster (start\_CPU - end\_CPU) and N is the number of worker threads (1 thread for each CPU core)
	\item Execute the benchmarks on a specific set of CPUs: \begin{verbatim}taskset -c cpu_list\end{verbatim}
\end{itemize}

\section{Design and Implementation}

\subsection{Benchmarks}

To investigate this issue, we selected 6 floating point benchmarks from \href{http://www.mrtc.mdh.se/projects/wcet/benchmarks.html}{mrtc.mdh.se}.
These are the selected benchmarks:
\begin{enumerate}
	\item ludcmp: implementation of the \href{https://en.wikipedia.org/wiki/LU_decomposition}{LU decomposition} algorithm
	\item minver: inversion of a floating point matrix
	\item qsort: non-recursive implementation of quick sort algorithm
	\item qurt: root computation of quadratic equations
	\item select: function to select the ${N}^{th}$ largest number in a floating point array
	\item sqrt: square root function implemented by Taylor series
\end{enumerate}

Since there is no automatic way in C to convert floating point operations into fixed point, we have defined rules to convert floating point benchmarks into fixed points.
\begin{enumerate}
	\item Convert all the floating values into fixed point values in a fair way, keeping the same variable dimension:
	\begin{itemize}
		\item float $\rightarrow$ int32\_t	
		\item double $\rightarrow$ int64\_t
	\end{itemize}
	\item Multiply every floating point value by ${2}^{SHIFT\_AMOUNT}$.
	\item Convert every multiplication and division into the corresponding fixed point function
	\begin{itemize}
		\item multiplication $\rightarrow$ fixed\_mul\_64bit	
		\item division $\rightarrow$ fixed\_div\_64bit
	\end{itemize}
\end{enumerate}

Moreover, all benchmarks are executed on random generated numbers, except qurt which computes multiple times the inversion of a matrix. 
\subsection{Sampling of the benchmarks}

Original benchmarks have no built-in method to obtain the elapsed execution time. We have implemented and integrated in C a function which randomly generates arguments for the benchmark and measures the elapsed time.
Measurements are stored in a .txt files for future analysis.

This is an example of the main function:
\begin{itemize}
	\item srand() is used to generate random numbers, using a seed to keep track of generated numbers
	\item EXEC\_NUM is a constant which regulates the number of samples generated. In our case, we generate 100.000 samples.
	\item function\_to\_call(val) is the benchmark function with the argument.
	\item diff(timespec,timespec) is a support function which makes the difference between two time instants.
	\item CLOCK\_MONOTONIC\_RAW is used to get the time instant from the system without alterations (even from ntp)
	\item SHIFT\_AMOUNT is used to convert floating point values into fixed points, an higher SHIFT\_AMOUNT means an higher precision.
\end{itemize}
\begin{lstlisting}
int main()
{
	struct timespec start,end;
	FILE * fp;
	fp = fopen ("benchmark_results.txt","w");
	int64_t val;
	srand(5);
	for (int i=0; i< EXEC_NUM ; i++){
		//convert float to int64 only if the benchmark is fixed point, if not no conversion is done
		val = (int64_t)((((rand() % 10000) / 100) - 50) * pow(2,SHIFT_AMOUNT));
		
		clock_gettime(CLOCK_MONOTONIC_RAW, &start);
		function_to_call(val);
		clock_gettime(CLOCK_MONOTONIC_RAW, &end);
		
		fprintf (fp, "%lld\n",(long long)(diff(start,end).tv_sec * pow(10,9))+(long long)diff(start,end).tv_nsec);
	}
	
	fclose (fp);
	return 0;
}
\end{lstlisting}

\subsection{Fixed point operations}

To support fixed point operations we had to introduce a shift amount, so that floating point numbers can be represented as integers with a fixed precision, given by the shift amount.\newline
Since we have analysed different architectures we had to take into account the limitations of the selected architectures.
x86\_64 have the support for 128-bit variables, which are necessary to compute multiplications and divisions. Instead, ARM architectures have no support for 128-bit variables, so we implemented a support library to handle this discrepancy.
The library automatically select the operations by checking if the architecture support 128-bit integers, if there is no support, the benchmark will use a portable implementation of multiplication and division which are based on 64-bit integers.\newline
Our implementation of fixed division and multiplication works with a variable shift amount, in our benchmarks we have always used a shift amount of 30 because in our scenario it allows to have the necessary precision.\newline
Note: to guarantee fairness we use hard-coded values for constants in benchmarks that are shifted by 30-bits, even though operations supports dynamical shift amounts these values have to be modified to comply with the new selected shift amount.\newline

\textbf{fixed\_div for 128-bit support}

\begin{lstlisting}
int64_t fixed_div_64(int64_t x, int64_t y, int shift_amount)
{
    return ((((__int128)x << shift_amount) / y));
}
\end{lstlisting}

\textbf{fixed\_mul for 128-bit support}

\begin{lstlisting}
int64_t fixed_mul_64(int64_t x, int64_t y, int shift_amount)
{
	return (int64_t)((((__int128)x * (__int128)y)) >> shift_amount);
}
\end{lstlisting}

\textbf{fixed\_div without 128-bit support}\newline
To implement the divison mantaining the 128 bit precision needed during the divison operation,  the dividend is splitted into two 64 bit integers. Then the divison is perfomed using the technique that can be found at this \href{https://codereview.stackexchange.com/questions/67962/mostly-portable-128-by-64-bit-division}{link}.

\textbf{fixed\_mul without 128-bit support}\newline
To implement the multiplication mantaining the 128 bit precision needed during the multiplication operation,  the two factors are splitted into two 64 bit integers each. Then the multiplication is perfomed using the technique that can be found at this \href{https://stackoverflow.com/questions/31652875/fastest-way-to-multiply-two-64-bit-ints-to-128-bit-then-to-64-bit}{link}.



\subsection{Statistics used}
To compare floating point results to fixed point results we computed through a Python script 8 metrics useful to analyse the predictability of a benchmark.

\begin{enumerate}
	\item Minimum execution time
	\item Maximum execution time
	\item Sample mean, which is computed as: $$\xoverline[0.8]{x} = \frac{X_1 + X_2 + \cdots + X_n}{n}
	= \frac{1}{n}\sum_{i}^{n} X_i$$
	\item Sample variance, which is computed as:$${s}^{2} = \frac{\sqrt{\sum_{i}^{n} {(x_i - \xoverline[0.8]{x})}^{2}}}{n-1}$$
	\item Sample standard deviation, which is computed as: 
	$$s = \frac{\sqrt{\sum_{i}^{n} (x_i - \xoverline[0.8]{x})}}{n-1}$$
	\item KPSS test statistic, which is computed as:$$S = \frac{{T}^{-2}\sum_{t=1}^{T} {\hat{S}_t}^{2}}{\sigma_\epsilon}$$
	Where:
	\begin{itemize}
		\item T is the sample size
		\item ${\sigma_\epsilon}$ is the long-run variance
		\item ${\hat{S}_t} = \sum_{i}^{t}e(t)$ is the partial sum of the errors of the regression $y(t)$
	\end{itemize}
	\item BDS test statistic, which is computed as: $$BDS_{\epsilon,m} = \frac{\sqrt{N}[C_{\epsilon,m}-{(C_{\epsilon,1})}^{m}]}{\sqrt{V_{\epsilon,m}}}$$
	Where:
	\begin{itemize}
		\item $C_{\epsilon,m} = \frac{1}{N_m(N_m-1)}\sum_{i\not=j}{I_{i,j;\epsilon}}$\\[0.1cm]
		Where:
		\begin{itemize}
			\item m is the number of embedding dimensions, used to embed the time series into m-dimensional vectors.
			\item ${I_{i,j;\epsilon}} = 1 \;\; if ||{x}^{m}_i - {x}^{m}_j|| <= \epsilon  \;\; = 0 \;\; otherwise$
		\end{itemize}
		
		\item $V_{\epsilon,m} = 4[{K}^{m} + 2\sum_{j=1}^{m-1}{K}^{m-j}{C}^{2j}_\epsilon + {(m-1)}^{2}{C}^{2m}_\epsilon - {m}^{2}K{C}^{2m-2}_\epsilon]$
		Where:
		\begin{itemize}
			\item $K=K_\epsilon = \frac{6}{N_m(N_m-1)*(N_m-2)}\sum_{i<j<N}h_{i,j,N;\epsilon}$
			\item $h_{i,j,N;\epsilon} = \frac{[I_{i,j;\epsilon}I_{j,N;\epsilon}+I_{i,N;\epsilon}I_{N,j;\epsilon}+I_{j,i;\epsilon}I_{i,N;\epsilon}]}{3}$
		\end{itemize}
	\end{itemize}
	We know under some hypothesis, that the quantity $[C_{\epsilon,m}-{(C_{\epsilon,1})}^{m}]$ can be considered as an asymptotic normal distribution with zero mean and variance $V_{\epsilon,m}$
	\item Hurst exponent, which is computed as:
	$\operatorname{E} \left [ \frac{R(n)}{S(n)} \right ]=C n^H  \text{  as } n \to \infty  \, $
	Where:
	\begin{itemize}
		\item $R(n)$ is the range of the first n cumulative deviations from the mean, and $s(n)$ is their standard deviation
		\item $\operatorname{E} \left [x \right ] \,$ is the expected value
		\item n is the time span of the observation (number of data points in a time series)
		\item C is a constant.
	\end{itemize}
\end{enumerate}

We are interested in execution time predictability, so we need to compare how much the measurements are spread through sample standard deviation and sample variance.
\href{http://debis.deu.edu.tr/userweb//onder.hanedar/dosyalar/kpss.pdf}{KPSS} test statistic is a one tailed test, which indicates if the time series is stationary around a mean or a linear trend. A stationary time series have constant statistical properties (like mean and variance) over time.
\begin{itemize}
	\item Null hypothesis: the data is stationary
	\item Alternative hypothesis: the data is not stationary
\end{itemize}
Instead, \href{https://www.researchgate.net/publication/46554708_A_Fast_Algorithm_for_the_BDS_Statistic}{BDS} statistic is a double tailed test, which indicates if the time series is serial dependent.
\begin{itemize}
 	\item Null hypothesis: the data is independently and identically distributed (I.I.D.)
 	\item Alternative hypothesis: the data is not I.I.D.
\end{itemize}

Since BDS test statistic complexity increase with the sample size, it can't be done on all the samples at once, instead we compute it using a sliding window on the samples and then select the maximum BDS obtained.



\section{Experimental Results}

\section{Conclusions}

\end{document} to your LaTeX file where you want your
% title page.
%
%%%%%%%%%%%%%%%%%%%%%%%%%%%%%%%%%%%%%%%%%
%\title{Title page with logo}
%----------------------------------------------------------------------------------------
%	PACKAGES AND OTHER DOCUMENT CONFIGURATIONS
%----------------------------------------------------------------------------------------

\documentclass[12pt]{article}
\usepackage[english]{babel}
\usepackage[utf8x]{inputenc}
\usepackage{amsmath}
\usepackage{graphicx}
\usepackage{listings}
\usepackage{color}
\usepackage{xcolor}
\usepackage{amssymb}
\usepackage{hyperref}

\definecolor{dkgreen}{rgb}{0,0.6,0}
\definecolor{gray}{rgb}{0.5,0.5,0.5}
\definecolor{mauve}{rgb}{0.58,0,0.82}

\lstset{frame=tb,
	language=C,
	aboveskip=3mm,
	belowskip=3mm,
	showstringspaces=false,
	columns=flexible,
	basicstyle={\small\ttfamily},
	numbers=none,
	numberstyle=\tiny\color{gray},
	keywordstyle=\color{blue},
	commentstyle=\color{dkgreen},
	stringstyle=\color{mauve},
	breaklines=true,
	breakatwhitespace=true,
	tabsize=3
}

\makeatletter
\newsavebox\myboxA
\newsavebox\myboxB
\newlength\mylenA

\newcommand*\xoverline[2][0.75]{%
	\sbox{\myboxA}{$\m@th#2$}%
	\setbox\myboxB\null% Phantom box
	\ht\myboxB=\ht\myboxA%
	\dp\myboxB=\dp\myboxA%
	\wd\myboxB=#1\wd\myboxA% Scale phantom
	\sbox\myboxB{$\m@th\overline{\copy\myboxB}$}%  Overlined phantom
	\setlength\mylenA{\the\wd\myboxA}%   calc width diff
	\addtolength\mylenA{-\the\wd\myboxB}%
	\ifdim\wd\myboxB<\wd\myboxA%
	\rlap{\hskip 0.5\mylenA\usebox\myboxB}{\usebox\myboxA}%
	\else
	\hskip -0.5\mylenA\rlap{\usebox\myboxA}{\hskip 0.5\mylenA\usebox\myboxB}%
	\fi}
\makeatother

\begin{document}

\begin{titlepage}

\newcommand{\HRule}{\rule{\linewidth}{0.5mm}} % Defines a new command for the horizontal lines, change thickness here

\center % Center everything on the page
 
%----------------------------------------------------------------------------------------
%	HEADING SECTIONS
%----------------------------------------------------------------------------------------

\textsc{\LARGE Politecnico di Milano}\\[1.5cm] % Name of your university/college
\textsc{\Large Dipartimento Elettronica, Informazione e Bioingegneria}\\[0.5cm] % Major heading such as course name
\textsc{\large HEAPLab Project Report}\\[0.5cm] % Minor heading such as course title

%----------------------------------------------------------------------------------------
%	TITLE SECTION
%----------------------------------------------------------------------------------------

\HRule \\[0.4cm]
{ \huge \bfseries Floating Point Predictability}\\[0.4cm] % Title of your document
\HRule \\[1.5cm]
 
%----------------------------------------------------------------------------------------
%	AUTHOR SECTION
%----------------------------------------------------------------------------------------

\begin{minipage}{0.4\textwidth}
\begin{flushleft} \large
\emph{Authors:}\\
Alessio \textsc{Cantina}\\ % Your name
Simone \textsc{Crippa}
\end{flushleft}
\end{minipage}
~
\begin{minipage}{0.5\textwidth}
\begin{flushright} \large
\emph{Supervisor:} \\
Dr. Federico \textsc{Reghenzani} % Supervisor's Name
\end{flushright}
\end{minipage}\\[1cm]

% If you don't want a supervisor, uncomment the two lines below and remove the section above
%\Large \emph{Author:}\\
%John \textsc{Smith}\\[3cm] % Your name

%----------------------------------------------------------------------------------------
%	DATE SECTION
%----------------------------------------------------------------------------------------

{\large \today}\\[2cm] % Date, change the \today to a set date if you want to be precise

%----------------------------------------------------------------------------------------
%	LOGO SECTION
%----------------------------------------------------------------------------------------

\includegraphics[width=100pt]{heaplogo.pdf}\\[1cm] % Include a department/university logo - this will require the graphicx package
 
%----------------------------------------------------------------------------------------

\vfill % Fill the rest of the page with whitespace

\end{titlepage}




\begin{abstract}
\textcolor{red}{Your abstract. - Summarize your work.}

This project aims to investigate floating point vs fixed point predictability.

To achieve this result, we have used 6 Worst Case Execution Time (WCET) benchmarks which involve a large number of floating point instructions.
We converted these benchmarks using only fixed point variables and then we added the logic to obtain multiple measurements.

Finally, we computed and analysed 8 statistics to find out if floating point operations or fixed point operations are more predictable.

\end{abstract}

\section{Introduction}

\textcolor{red}{Your introduction goes here! Some examples of commonly used commands and features are listed below, to help you get started.}

In this paragraph we describe all the commands used to process the benchmarks.

\subsection{Devices used}

To have a better view of the problem, we decided to get samples from devices with different architecture.
In this project, we considered 3 architectures and 5 devices.\newline 
armv7l devices:
\begin{itemize}
	\item Device name: Freescale i.MX 6 Quad. Processor configuration: 4x Cortex-A9 @ 1.2 GHz
	\item Device name: ODROID-XU3. Processor name: Exynos 5422. Processor configuration: big.LITTLE 8 cores, 4x Cortex-A15 @ 2.0 GHz, 4x Cortex-A7 @ 1.4 GHz 
\end{itemize}
armv8a devices:
\begin{itemize}
	\item Device name: Xiaomi Mi 5s. Processor name: Snapdragon 821 (MSM8996). Processor configuration: big.LITTLE 4 cores: 2x Kryo @ 2.15 GHz, 2x Kryo @ 1.6 GHz
	\item Device name: Xiaomi Mi Mix 2. Processor name: Snapdragon 835 (MSM8998). Processor configuration: big.LITTLE 8 cores: 4x Kryo @ 2.45 GHz, 4x Kryo @ 1.9 GHz
\end{itemize}
x86\_64 devices:
\begin{itemize}
	\item Device name: Dell XPS 13 9360. Processor name: Intel i5 8250u. Processor configuration: 4 cores (8 threads) @ 3.40 GHz
\end{itemize}
\subsection{Compiling commands}

\begin{itemize}
	\item Compile a floating point benchmark with the support library: \begin{verbatim}gcc -o "output_name" "benchmark_name"\end{verbatim}
	\item Cross-compile a benchmark, useful to execute a benchmark on a smartphone with ARM processor: \begin{verbatim}arm-linux-gnueabi-gcc -static -march="arm_platform" -o 
	"output_name" "benchmark_name"\end{verbatim}
	\item To compile a fixed point benchmark remember to append at the end of the command: \begin{verbatim}fixed_op_64bit.c\end{verbatim}
\end{itemize}


\subsection{make commands}

A makefile has been produced so that make can be used to compile and execute all the benchmarks at once:

\begin{itemize}
	\item Initialize the project directory with support folders:\begin{verbatim}make init\end{verbatim}
	\item Compile all the benchmarks: \begin{verbatim}make\end{verbatim}
	\item Execute all the benchmarks: \begin{verbatim}make run\end{verbatim}
	\item Clear the project directory: \begin{verbatim}make clean\end{verbatim}
\end{itemize}

Note: the makefile supports gcc compilation as is, but it needs to be modified for cross compilation. Detailed instructions are available inside the makefile.

\subsection{Additional commands}

These commands are used to run benchmarks with a stress test running on the CPU cache

\begin{itemize}
	\item Stress the CPU cache on a specific set of CPUs:\begin{verbatim}taskset -c cpu_list stress-ng --cache N \end{verbatim} where cpu\_list is the select cluster (start\_CPU - end\_CPU) and N is the number of worker threads (1 thread for each CPU core)
	\item Execute the benchmarks on a specific set of CPUs: \begin{verbatim}taskset -c cpu_list\end{verbatim}
\end{itemize}

\section{Design and Implementation}

\subsection{Benchmarks}

To investigate this issue, we selected 6 floating point benchmarks from \href{http://www.mrtc.mdh.se/projects/wcet/benchmarks.html}{mrtc.mdh.se}.
These are the selected benchmarks:
\begin{enumerate}
	\item ludcmp: implementation of the \href{https://en.wikipedia.org/wiki/LU_decomposition}{LU decomposition} algorithm
	\item minver: inversion of a floating point matrix
	\item qsort: non-recursive implementation of quick sort algorithm
	\item qurt: root computation of quadratic equations
	\item select: function to select the ${N}^{th}$ largest number in a floating point array
	\item sqrt: square root function implemented by Taylor series
\end{enumerate}

Since there is no automatic way in C to convert floating point operations into fixed point, we have defined rules to convert floating point benchmarks into fixed points.
\begin{enumerate}
	\item Convert all the floating values into fixed point values in a fair way, keeping the same variable dimension:
	\begin{itemize}
		\item float $\rightarrow$ int32\_t	
		\item double $\rightarrow$ int64\_t
	\end{itemize}
	\item Multiply every floating point value by ${2}^{SHIFT\_AMOUNT}$.
	\item Convert every multiplication and division into the corresponding fixed point function
	\begin{itemize}
		\item multiplication $\rightarrow$ fixed\_mul\_64bit	
		\item division $\rightarrow$ fixed\_div\_64bit
	\end{itemize}
\end{enumerate}

Moreover, all benchmarks are executed on random generated numbers, except qurt which computes multiple times the inversion of a matrix. 
\subsection{Sampling of the benchmarks}

Original benchmarks have no built-in method to obtain the elapsed execution time. We have implemented and integrated in C a function which randomly generates arguments for the benchmark and measures the elapsed time.
Measurements are stored in a .txt files for future analysis.

This is an example of the main function:
\begin{itemize}
	\item srand() is used to generate random numbers, using a seed to keep track of generated numbers
	\item EXEC\_NUM is a constant which regulates the number of samples generated. In our case, we generate 100.000 samples.
	\item function\_to\_call(val) is the benchmark function with the argument.
	\item diff(timespec,timespec) is a support function which makes the difference between two time instants.
	\item CLOCK\_MONOTONIC\_RAW is used to get the time instant from the system without alterations (even from ntp)
	\item SHIFT\_AMOUNT is used to convert floating point values into fixed points, an higher SHIFT\_AMOUNT means an higher precision.
\end{itemize}
\begin{lstlisting}
int main()
{
	struct timespec start,end;
	FILE * fp;
	fp = fopen ("benchmark_results.txt","w");
	int64_t val;
	srand(5);
	for (int i=0; i< EXEC_NUM ; i++){
		//convert float to int64 only if the benchmark is fixed point, if not no conversion is done
		val = (int64_t)((((rand() % 10000) / 100) - 50) * pow(2,SHIFT_AMOUNT));
		
		clock_gettime(CLOCK_MONOTONIC_RAW, &start);
		function_to_call(val);
		clock_gettime(CLOCK_MONOTONIC_RAW, &end);
		
		fprintf (fp, "%lld\n",(long long)(diff(start,end).tv_sec * pow(10,9))+(long long)diff(start,end).tv_nsec);
	}
	
	fclose (fp);
	return 0;
}
\end{lstlisting}

\subsection{Fixed point operations}

To support fixed point operations we had to introduce a shift amount, so that floating point numbers can be represented as integers with a fixed precision, given by the shift amount.\newline
Since we have analysed different architectures we had to take into account the limitations of the selected architectures.
x86\_64 have the support for 128-bit variables, which are necessary to compute multiplications and divisions. Instead, ARM architectures have no support for 128-bit variables, so we implemented a support library to handle this discrepancy.
The library automatically select the operations by checking if the architecture support 128-bit integers, if there is no support, the benchmark will use a portable implementation of multiplication and division which are based on 64-bit integers.\newline
Our implementation of fixed division and multiplication works with a variable shift amount, in our benchmarks we have always used a shift amount of 30 because in our scenario it allows to have the necessary precision.\newline
Note: to guarantee fairness we use hard-coded values for constants in benchmarks that are shifted by 30-bits, even though operations supports dynamical shift amounts these values have to be modified to comply with the new selected shift amount.\newline

\textbf{fixed\_div for 128-bit support}

\begin{lstlisting}
int64_t fixed_div_64(int64_t x, int64_t y, int shift_amount)
{
    return ((((__int128)x << shift_amount) / y));
}
\end{lstlisting}

\textbf{fixed\_mul for 128-bit support}

\begin{lstlisting}
int64_t fixed_mul_64(int64_t x, int64_t y, int shift_amount)
{
	return (int64_t)((((__int128)x * (__int128)y)) >> shift_amount);
}
\end{lstlisting}

\textbf{fixed\_div without 128-bit support}\newline
To implement the divison mantaining the 128 bit precision needed during the divison operation,  the dividend is splitted into two 64 bit integers. Then the divison is perfomed using the technique that can be found at this \href{https://codereview.stackexchange.com/questions/67962/mostly-portable-128-by-64-bit-division}{link}.

\textbf{fixed\_mul without 128-bit support}\newline
To implement the multiplication mantaining the 128 bit precision needed during the multiplication operation,  the two factors are splitted into two 64 bit integers each. Then the multiplication is perfomed using the technique that can be found at this \href{https://stackoverflow.com/questions/31652875/fastest-way-to-multiply-two-64-bit-ints-to-128-bit-then-to-64-bit}{link}.



\subsection{Statistics used}
To compare floating point results to fixed point results we computed through a Python script 8 metrics useful to analyse the predictability of a benchmark.

\begin{enumerate}
	\item Minimum execution time
	\item Maximum execution time
	\item Sample mean, which is computed as: $$\xoverline[0.8]{x} = \frac{X_1 + X_2 + \cdots + X_n}{n}
	= \frac{1}{n}\sum_{i}^{n} X_i$$
	\item Sample variance, which is computed as:$${s}^{2} = \frac{\sqrt{\sum_{i}^{n} {(x_i - \xoverline[0.8]{x})}^{2}}}{n-1}$$
	\item Sample standard deviation, which is computed as: 
	$$s = \frac{\sqrt{\sum_{i}^{n} (x_i - \xoverline[0.8]{x})}}{n-1}$$
	\item KPSS test statistic, which is computed as:$$S = \frac{{T}^{-2}\sum_{t=1}^{T} {\hat{S}_t}^{2}}{\sigma_\epsilon}$$
	Where:
	\begin{itemize}
		\item T is the sample size
		\item ${\sigma_\epsilon}$ is the long-run variance
		\item ${\hat{S}_t} = \sum_{i}^{t}e(t)$ is the partial sum of the errors of the regression $y(t)$
	\end{itemize}
	\item BDS test statistic, which is computed as: $$BDS_{\epsilon,m} = \frac{\sqrt{N}[C_{\epsilon,m}-{(C_{\epsilon,1})}^{m}]}{\sqrt{V_{\epsilon,m}}}$$
	Where:
	\begin{itemize}
		\item $C_{\epsilon,m} = \frac{1}{N_m(N_m-1)}\sum_{i\not=j}{I_{i,j;\epsilon}}$\\[0.1cm]
		Where:
		\begin{itemize}
			\item m is the number of embedding dimensions, used to embed the time series into m-dimensional vectors.
			\item ${I_{i,j;\epsilon}} = 1 \;\; if ||{x}^{m}_i - {x}^{m}_j|| <= \epsilon  \;\; = 0 \;\; otherwise$
		\end{itemize}
		
		\item $V_{\epsilon,m} = 4[{K}^{m} + 2\sum_{j=1}^{m-1}{K}^{m-j}{C}^{2j}_\epsilon + {(m-1)}^{2}{C}^{2m}_\epsilon - {m}^{2}K{C}^{2m-2}_\epsilon]$
		Where:
		\begin{itemize}
			\item $K=K_\epsilon = \frac{6}{N_m(N_m-1)*(N_m-2)}\sum_{i<j<N}h_{i,j,N;\epsilon}$
			\item $h_{i,j,N;\epsilon} = \frac{[I_{i,j;\epsilon}I_{j,N;\epsilon}+I_{i,N;\epsilon}I_{N,j;\epsilon}+I_{j,i;\epsilon}I_{i,N;\epsilon}]}{3}$
		\end{itemize}
	\end{itemize}
	We know under some hypothesis, that the quantity $[C_{\epsilon,m}-{(C_{\epsilon,1})}^{m}]$ can be considered as an asymptotic normal distribution with zero mean and variance $V_{\epsilon,m}$
	\item Hurst exponent, which is computed as:
	$\operatorname{E} \left [ \frac{R(n)}{S(n)} \right ]=C n^H  \text{  as } n \to \infty  \, $
	Where:
	\begin{itemize}
		\item $R(n)$ is the range of the first n cumulative deviations from the mean, and $s(n)$ is their standard deviation
		\item $\operatorname{E} \left [x \right ] \,$ is the expected value
		\item n is the time span of the observation (number of data points in a time series)
		\item C is a constant.
	\end{itemize}
\end{enumerate}

We are interested in execution time predictability, so we need to compare how much the measurements are spread through sample standard deviation and sample variance.
\href{http://debis.deu.edu.tr/userweb//onder.hanedar/dosyalar/kpss.pdf}{KPSS} test statistic is a one tailed test, which indicates if the time series is stationary around a mean or a linear trend. A stationary time series have constant statistical properties (like mean and variance) over time.
\begin{itemize}
	\item Null hypothesis: the data is stationary
	\item Alternative hypothesis: the data is not stationary
\end{itemize}
Instead, \href{https://www.researchgate.net/publication/46554708_A_Fast_Algorithm_for_the_BDS_Statistic}{BDS} statistic is a double tailed test, which indicates if the time series is serial dependent.
\begin{itemize}
 	\item Null hypothesis: the data is independently and identically distributed (I.I.D.)
 	\item Alternative hypothesis: the data is not I.I.D.
\end{itemize}

Since BDS test statistic complexity increase with the sample size, it can't be done on all the samples at once, instead we compute it using a sliding window on the samples and then select the maximum BDS obtained.



\section{Experimental Results}

\section{Conclusions}

\end{document} to your LaTeX file where you want your
% title page.
%
%%%%%%%%%%%%%%%%%%%%%%%%%%%%%%%%%%%%%%%%%
%\title{Title page with logo}
%----------------------------------------------------------------------------------------
%	PACKAGES AND OTHER DOCUMENT CONFIGURATIONS
%----------------------------------------------------------------------------------------

\documentclass[12pt]{article}
\usepackage[english]{babel}
\usepackage[utf8x]{inputenc}
\usepackage{amsmath}
\usepackage{graphicx}
\usepackage{listings}
\usepackage{color}
\usepackage{xcolor}
\usepackage{amssymb}
\usepackage{hyperref}

\definecolor{dkgreen}{rgb}{0,0.6,0}
\definecolor{gray}{rgb}{0.5,0.5,0.5}
\definecolor{mauve}{rgb}{0.58,0,0.82}

\lstset{frame=tb,
	language=C,
	aboveskip=3mm,
	belowskip=3mm,
	showstringspaces=false,
	columns=flexible,
	basicstyle={\small\ttfamily},
	numbers=none,
	numberstyle=\tiny\color{gray},
	keywordstyle=\color{blue},
	commentstyle=\color{dkgreen},
	stringstyle=\color{mauve},
	breaklines=true,
	breakatwhitespace=true,
	tabsize=3
}

\makeatletter
\newsavebox\myboxA
\newsavebox\myboxB
\newlength\mylenA

\newcommand*\xoverline[2][0.75]{%
	\sbox{\myboxA}{$\m@th#2$}%
	\setbox\myboxB\null% Phantom box
	\ht\myboxB=\ht\myboxA%
	\dp\myboxB=\dp\myboxA%
	\wd\myboxB=#1\wd\myboxA% Scale phantom
	\sbox\myboxB{$\m@th\overline{\copy\myboxB}$}%  Overlined phantom
	\setlength\mylenA{\the\wd\myboxA}%   calc width diff
	\addtolength\mylenA{-\the\wd\myboxB}%
	\ifdim\wd\myboxB<\wd\myboxA%
	\rlap{\hskip 0.5\mylenA\usebox\myboxB}{\usebox\myboxA}%
	\else
	\hskip -0.5\mylenA\rlap{\usebox\myboxA}{\hskip 0.5\mylenA\usebox\myboxB}%
	\fi}
\makeatother

\begin{document}

\begin{titlepage}

\newcommand{\HRule}{\rule{\linewidth}{0.5mm}} % Defines a new command for the horizontal lines, change thickness here

\center % Center everything on the page
 
%----------------------------------------------------------------------------------------
%	HEADING SECTIONS
%----------------------------------------------------------------------------------------

\textsc{\LARGE Politecnico di Milano}\\[1.5cm] % Name of your university/college
\textsc{\Large Dipartimento Elettronica, Informazione e Bioingegneria}\\[0.5cm] % Major heading such as course name
\textsc{\large HEAPLab Project Report}\\[0.5cm] % Minor heading such as course title

%----------------------------------------------------------------------------------------
%	TITLE SECTION
%----------------------------------------------------------------------------------------

\HRule \\[0.4cm]
{ \huge \bfseries Floating Point Predictability}\\[0.4cm] % Title of your document
\HRule \\[1.5cm]
 
%----------------------------------------------------------------------------------------
%	AUTHOR SECTION
%----------------------------------------------------------------------------------------

\begin{minipage}{0.4\textwidth}
\begin{flushleft} \large
\emph{Authors:}\\
Alessio \textsc{Cantina}\\ % Your name
Simone \textsc{Crippa}
\end{flushleft}
\end{minipage}
~
\begin{minipage}{0.5\textwidth}
\begin{flushright} \large
\emph{Supervisor:} \\
Dr. Federico \textsc{Reghenzani} % Supervisor's Name
\end{flushright}
\end{minipage}\\[1cm]

% If you don't want a supervisor, uncomment the two lines below and remove the section above
%\Large \emph{Author:}\\
%John \textsc{Smith}\\[3cm] % Your name

%----------------------------------------------------------------------------------------
%	DATE SECTION
%----------------------------------------------------------------------------------------

{\large \today}\\[2cm] % Date, change the \today to a set date if you want to be precise

%----------------------------------------------------------------------------------------
%	LOGO SECTION
%----------------------------------------------------------------------------------------

\includegraphics[width=100pt]{heaplogo.pdf}\\[1cm] % Include a department/university logo - this will require the graphicx package
 
%----------------------------------------------------------------------------------------

\vfill % Fill the rest of the page with whitespace

\end{titlepage}




\begin{abstract}
\textcolor{red}{Your abstract. - Summarize your work.}

This project aims to investigate floating point vs fixed point predictability.

To achieve this result, we have used 6 Worst Case Execution Time (WCET) benchmarks which involve a large number of floating point instructions.
We converted these benchmarks using only fixed point variables and then we added the logic to obtain multiple measurements.

Finally, we computed and analysed 8 statistics to find out if floating point operations or fixed point operations are more predictable.

\end{abstract}

\section{Introduction}

\textcolor{red}{Your introduction goes here! Some examples of commonly used commands and features are listed below, to help you get started.}

In this paragraph we describe all the commands used to process the benchmarks.

\subsection{Devices used}

To have a better view of the problem, we decided to get samples from devices with different architecture.
In this project, we considered 3 architectures and 5 devices.\newline 
armv7l devices:
\begin{itemize}
	\item Device name: Freescale i.MX 6 Quad. Processor configuration: 4x Cortex-A9 @ 1.2 GHz
	\item Device name: ODROID-XU3. Processor name: Exynos 5422. Processor configuration: big.LITTLE 8 cores, 4x Cortex-A15 @ 2.0 GHz, 4x Cortex-A7 @ 1.4 GHz 
\end{itemize}
armv8a devices:
\begin{itemize}
	\item Device name: Xiaomi Mi 5s. Processor name: Snapdragon 821 (MSM8996). Processor configuration: big.LITTLE 4 cores: 2x Kryo @ 2.15 GHz, 2x Kryo @ 1.6 GHz
	\item Device name: Xiaomi Mi Mix 2. Processor name: Snapdragon 835 (MSM8998). Processor configuration: big.LITTLE 8 cores: 4x Kryo @ 2.45 GHz, 4x Kryo @ 1.9 GHz
\end{itemize}
x86\_64 devices:
\begin{itemize}
	\item Device name: Dell XPS 13 9360. Processor name: Intel i5 8250u. Processor configuration: 4 cores (8 threads) @ 3.40 GHz
\end{itemize}
\subsection{Compiling commands}

\begin{itemize}
	\item Compile a floating point benchmark with the support library: \begin{verbatim}gcc -o "output_name" "benchmark_name"\end{verbatim}
	\item Cross-compile a benchmark, useful to execute a benchmark on a smartphone with ARM processor: \begin{verbatim}arm-linux-gnueabi-gcc -static -march="arm_platform" -o 
	"output_name" "benchmark_name"\end{verbatim}
	\item To compile a fixed point benchmark remember to append at the end of the command: \begin{verbatim}fixed_op_64bit.c\end{verbatim}
\end{itemize}


\subsection{make commands}

A makefile has been produced so that make can be used to compile and execute all the benchmarks at once:

\begin{itemize}
	\item Initialize the project directory with support folders:\begin{verbatim}make init\end{verbatim}
	\item Compile all the benchmarks: \begin{verbatim}make\end{verbatim}
	\item Execute all the benchmarks: \begin{verbatim}make run\end{verbatim}
	\item Clear the project directory: \begin{verbatim}make clean\end{verbatim}
\end{itemize}

Note: the makefile supports gcc compilation as is, but it needs to be modified for cross compilation. Detailed instructions are available inside the makefile.

\subsection{Additional commands}

These commands are used to run benchmarks with a stress test running on the CPU cache

\begin{itemize}
	\item Stress the CPU cache on a specific set of CPUs:\begin{verbatim}taskset -c cpu_list stress-ng --cache N \end{verbatim} where cpu\_list is the select cluster (start\_CPU - end\_CPU) and N is the number of worker threads (1 thread for each CPU core)
	\item Execute the benchmarks on a specific set of CPUs: \begin{verbatim}taskset -c cpu_list\end{verbatim}
\end{itemize}

\section{Design and Implementation}

\subsection{Benchmarks}

To investigate this issue, we selected 6 floating point benchmarks from \href{http://www.mrtc.mdh.se/projects/wcet/benchmarks.html}{mrtc.mdh.se}.
These are the selected benchmarks:
\begin{enumerate}
	\item ludcmp: implementation of the \href{https://en.wikipedia.org/wiki/LU_decomposition}{LU decomposition} algorithm
	\item minver: inversion of a floating point matrix
	\item qsort: non-recursive implementation of quick sort algorithm
	\item qurt: root computation of quadratic equations
	\item select: function to select the ${N}^{th}$ largest number in a floating point array
	\item sqrt: square root function implemented by Taylor series
\end{enumerate}

Since there is no automatic way in C to convert floating point operations into fixed point, we have defined rules to convert floating point benchmarks into fixed points.
\begin{enumerate}
	\item Convert all the floating values into fixed point values in a fair way, keeping the same variable dimension:
	\begin{itemize}
		\item float $\rightarrow$ int32\_t	
		\item double $\rightarrow$ int64\_t
	\end{itemize}
	\item Multiply every floating point value by ${2}^{SHIFT\_AMOUNT}$.
	\item Convert every multiplication and division into the corresponding fixed point function
	\begin{itemize}
		\item multiplication $\rightarrow$ fixed\_mul\_64bit	
		\item division $\rightarrow$ fixed\_div\_64bit
	\end{itemize}
\end{enumerate}

Moreover, all benchmarks are executed on random generated numbers, except qurt which computes multiple times the inversion of a matrix. 
\subsection{Sampling of the benchmarks}

Original benchmarks have no built-in method to obtain the elapsed execution time. We have implemented and integrated in C a function which randomly generates arguments for the benchmark and measures the elapsed time.
Measurements are stored in a .txt files for future analysis.

This is an example of the main function:
\begin{itemize}
	\item srand() is used to generate random numbers, using a seed to keep track of generated numbers
	\item EXEC\_NUM is a constant which regulates the number of samples generated. In our case, we generate 100.000 samples.
	\item function\_to\_call(val) is the benchmark function with the argument.
	\item diff(timespec,timespec) is a support function which makes the difference between two time instants.
	\item CLOCK\_MONOTONIC\_RAW is used to get the time instant from the system without alterations (even from ntp)
	\item SHIFT\_AMOUNT is used to convert floating point values into fixed points, an higher SHIFT\_AMOUNT means an higher precision.
\end{itemize}
\begin{lstlisting}
int main()
{
	struct timespec start,end;
	FILE * fp;
	fp = fopen ("benchmark_results.txt","w");
	int64_t val;
	srand(5);
	for (int i=0; i< EXEC_NUM ; i++){
		//convert float to int64 only if the benchmark is fixed point, if not no conversion is done
		val = (int64_t)((((rand() % 10000) / 100) - 50) * pow(2,SHIFT_AMOUNT));
		
		clock_gettime(CLOCK_MONOTONIC_RAW, &start);
		function_to_call(val);
		clock_gettime(CLOCK_MONOTONIC_RAW, &end);
		
		fprintf (fp, "%lld\n",(long long)(diff(start,end).tv_sec * pow(10,9))+(long long)diff(start,end).tv_nsec);
	}
	
	fclose (fp);
	return 0;
}
\end{lstlisting}

\subsection{Fixed point operations}

To support fixed point operations we had to introduce a shift amount, so that floating point numbers can be represented as integers with a fixed precision, given by the shift amount.\newline
Since we have analysed different architectures we had to take into account the limitations of the selected architectures.
x86\_64 have the support for 128-bit variables, which are necessary to compute multiplications and divisions. Instead, ARM architectures have no support for 128-bit variables, so we implemented a support library to handle this discrepancy.
The library automatically select the operations by checking if the architecture support 128-bit integers, if there is no support, the benchmark will use a portable implementation of multiplication and division which are based on 64-bit integers.\newline
Our implementation of fixed division and multiplication works with a variable shift amount, in our benchmarks we have always used a shift amount of 30 because in our scenario it allows to have the necessary precision.\newline
Note: to guarantee fairness we use hard-coded values for constants in benchmarks that are shifted by 30-bits, even though operations supports dynamical shift amounts these values have to be modified to comply with the new selected shift amount.\newline

\textbf{fixed\_div for 128-bit support}

\begin{lstlisting}
int64_t fixed_div_64(int64_t x, int64_t y, int shift_amount)
{
    return ((((__int128)x << shift_amount) / y));
}
\end{lstlisting}

\textbf{fixed\_mul for 128-bit support}

\begin{lstlisting}
int64_t fixed_mul_64(int64_t x, int64_t y, int shift_amount)
{
	return (int64_t)((((__int128)x * (__int128)y)) >> shift_amount);
}
\end{lstlisting}

\textbf{fixed\_div without 128-bit support}\newline
To implement the divison mantaining the 128 bit precision needed during the divison operation,  the dividend is splitted into two 64 bit integers. Then the divison is perfomed using the technique that can be found at this \href{https://codereview.stackexchange.com/questions/67962/mostly-portable-128-by-64-bit-division}{link}.

\textbf{fixed\_mul without 128-bit support}\newline
To implement the multiplication mantaining the 128 bit precision needed during the multiplication operation,  the two factors are splitted into two 64 bit integers each. Then the multiplication is perfomed using the technique that can be found at this \href{https://stackoverflow.com/questions/31652875/fastest-way-to-multiply-two-64-bit-ints-to-128-bit-then-to-64-bit}{link}.



\subsection{Statistics used}
To compare floating point results to fixed point results we computed through a Python script 8 metrics useful to analyse the predictability of a benchmark.

\begin{enumerate}
	\item Minimum execution time
	\item Maximum execution time
	\item Sample mean, which is computed as: $$\xoverline[0.8]{x} = \frac{X_1 + X_2 + \cdots + X_n}{n}
	= \frac{1}{n}\sum_{i}^{n} X_i$$
	\item Sample variance, which is computed as:$${s}^{2} = \frac{\sqrt{\sum_{i}^{n} {(x_i - \xoverline[0.8]{x})}^{2}}}{n-1}$$
	\item Sample standard deviation, which is computed as: 
	$$s = \frac{\sqrt{\sum_{i}^{n} (x_i - \xoverline[0.8]{x})}}{n-1}$$
	\item KPSS test statistic, which is computed as:$$S = \frac{{T}^{-2}\sum_{t=1}^{T} {\hat{S}_t}^{2}}{\sigma_\epsilon}$$
	Where:
	\begin{itemize}
		\item T is the sample size
		\item ${\sigma_\epsilon}$ is the long-run variance
		\item ${\hat{S}_t} = \sum_{i}^{t}e(t)$ is the partial sum of the errors of the regression $y(t)$
	\end{itemize}
	\item BDS test statistic, which is computed as: $$BDS_{\epsilon,m} = \frac{\sqrt{N}[C_{\epsilon,m}-{(C_{\epsilon,1})}^{m}]}{\sqrt{V_{\epsilon,m}}}$$
	Where:
	\begin{itemize}
		\item $C_{\epsilon,m} = \frac{1}{N_m(N_m-1)}\sum_{i\not=j}{I_{i,j;\epsilon}}$\\[0.1cm]
		Where:
		\begin{itemize}
			\item m is the number of embedding dimensions, used to embed the time series into m-dimensional vectors.
			\item ${I_{i,j;\epsilon}} = 1 \;\; if ||{x}^{m}_i - {x}^{m}_j|| <= \epsilon  \;\; = 0 \;\; otherwise$
		\end{itemize}
		
		\item $V_{\epsilon,m} = 4[{K}^{m} + 2\sum_{j=1}^{m-1}{K}^{m-j}{C}^{2j}_\epsilon + {(m-1)}^{2}{C}^{2m}_\epsilon - {m}^{2}K{C}^{2m-2}_\epsilon]$
		Where:
		\begin{itemize}
			\item $K=K_\epsilon = \frac{6}{N_m(N_m-1)*(N_m-2)}\sum_{i<j<N}h_{i,j,N;\epsilon}$
			\item $h_{i,j,N;\epsilon} = \frac{[I_{i,j;\epsilon}I_{j,N;\epsilon}+I_{i,N;\epsilon}I_{N,j;\epsilon}+I_{j,i;\epsilon}I_{i,N;\epsilon}]}{3}$
		\end{itemize}
	\end{itemize}
	We know under some hypothesis, that the quantity $[C_{\epsilon,m}-{(C_{\epsilon,1})}^{m}]$ can be considered as an asymptotic normal distribution with zero mean and variance $V_{\epsilon,m}$
	\item Hurst exponent, which is computed as:
	$\operatorname{E} \left [ \frac{R(n)}{S(n)} \right ]=C n^H  \text{  as } n \to \infty  \, $
	Where:
	\begin{itemize}
		\item $R(n)$ is the range of the first n cumulative deviations from the mean, and $s(n)$ is their standard deviation
		\item $\operatorname{E} \left [x \right ] \,$ is the expected value
		\item n is the time span of the observation (number of data points in a time series)
		\item C is a constant.
	\end{itemize}
\end{enumerate}

We are interested in execution time predictability, so we need to compare how much the measurements are spread through sample standard deviation and sample variance.
\href{http://debis.deu.edu.tr/userweb//onder.hanedar/dosyalar/kpss.pdf}{KPSS} test statistic is a one tailed test, which indicates if the time series is stationary around a mean or a linear trend. A stationary time series have constant statistical properties (like mean and variance) over time.
\begin{itemize}
	\item Null hypothesis: the data is stationary
	\item Alternative hypothesis: the data is not stationary
\end{itemize}
Instead, \href{https://www.researchgate.net/publication/46554708_A_Fast_Algorithm_for_the_BDS_Statistic}{BDS} statistic is a double tailed test, which indicates if the time series is serial dependent.
\begin{itemize}
 	\item Null hypothesis: the data is independently and identically distributed (I.I.D.)
 	\item Alternative hypothesis: the data is not I.I.D.
\end{itemize}

Since BDS test statistic complexity increase with the sample size, it can't be done on all the samples at once, instead we compute it using a sliding window on the samples and then select the maximum BDS obtained.



\section{Experimental Results}

\section{Conclusions}

\end{document}